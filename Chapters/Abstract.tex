% !TeX root = ../BA_LateXVorlage_v3-2.tex
\begin{center}
\renewcommand{\abstractname}{Titel}
\begin{abstract}
\noindent Entwicklung einer Software zum Vergleich textbasierter Dateien mit graphischer Anzeige der Unterschiede und besonderem Fokus auf XML und JSON
\end{abstract}

\renewcommand{\abstractname}{Zusammenfassung}
\begin{abstract}\label{abstract} 
\noindent
Wachsende Datenmengen sind ein stetiger Trend in der Informatik. Häufig können Anwendungsfälle gefunden werden bei denen es interessant ist die Unterschiede zwischen diesen Daten, bspw. bei der Versionierung, zu untersuchen. Im Rahmen dieser Bachelorarbeit wird eine Software erstellt, die beliebig viele textbasierte Dateien miteinander vergleicht, ihre Ähnlichkeit quantifiziert und die Möglichkeit bietet, Unterschiede für bis zu drei Dateien graphisch im Detail nachzuvollziehen. 
Weiterhin werden besondere Operationen für die gängigen Dateiformate XML und JSON implementiert um deren Inhalte sinnvoll vergleichbar zu machen. Zusätzlich werden für diese Dateiformate spezialisierte Vergleichsalgorithmen auf Basis ihrer Dokumentbäume entworfen und implementiert.

\vspace{1.0cm}
\noindent \textbf{Stichwörter: }Textvergleich, Diff, XML, JSON
 
\end{abstract}







\vspace{1.5cm}

\renewcommand{\abstractname}{Title}
\begin{abstract}
\noindent Development of a software to compare text-based files with graphical display of differences and special focus on XML and JSON
\end{abstract}

\renewcommand{\abstractname}{Abstract}
\begin{abstract}
\noindent
Growing amounts of data are a constant trend in computer science. Frequently, use cases can be found where it is interesting to look at the differences between parts of this data, for example in the case of versioning. In the context of this bachelor thesis a software is created, which compares arbitrarily many text-based files with each other, quantifies their similarity and offers the possibility to trace differences graphically in detail for up to three files. 
Furthermore, special operations for the common file formats XML and JSON are implemented to make their contents meaningfully comparable. Additionally, specialized comparison algorithms are designed and implemented for these file formats based on their document trees.


\vspace{1.0cm}
\noindent \textbf{Keywords: }text comparison, diff, XML, JSON
\end{abstract}



\end{center}